\documentclass[10pt,a4paper]{article}
\usepackage[utf8]{inputenc}
\usepackage[francais]{babel}
\usepackage[T1]{fontenc}
\usepackage{amsmath}
\usepackage{amsfonts}
\usepackage{amssymb}
\usepackage{graphicx}
\usepackage{epstopdf}
\usepackage{fancyhdr}
\pagestyle{fancy}
\usepackage{cite}


\renewcommand{\headrulewidth}{1pt}
\fancyhead[R]{} 
\fancyhead[L]{\textit{\leftmark}}

\addtolength{\hoffset}{-1.5cm}
\addtolength{\textwidth}{3.5cm}

\title{Projet MDI 343 \\
Systèmes de recommandation}
\author{Nicolas Keriven et Jean-Baptiste Alayrac}

\begin{document}
\maketitle

\hrulefill
\vspace{2cm}

% Si on veut mettre un abstract

%\renewcommand{\abstractname}{Résumé}
%\begin{abstract}
%
%
%\end{abstract}

% Figure 
%\begin{center}
%\begin{figure}[ht!]
%\includegraphics[width=\columnwidth]{fig/name.extension}
%\caption{\label{lab} titre}
%\end{figure}
%\end{center}

\newpage
\tableofcontents

\section*{Introduction}
\addcontentsline{toc}{section}{Introduction}
 

\newpage


\section{Présentation du problème et notations}

\subsection{Différentes méthodes}

\subsection{Factorisation de matrice}

\subsection{Notre approche et notations}

%% partie JB
%




\section{Algorithmes}

\subsection{Descente de gradient stochastique simple}

% Parler de la différence entre tirage avec remise ou sans. Quel est l'avantage théorique d'une descente de gradient sto?


\subsection{Algorithme parallélisé Jellyfish }





\section{Résultats}





\section*{Conclusion}

\newpage

\bibliographystyle{unsrt}
\bibliography{biblio}
\end{document}